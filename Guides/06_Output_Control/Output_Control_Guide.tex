% Options for packages loaded elsewhere
\PassOptionsToPackage{unicode}{hyperref}
\PassOptionsToPackage{hyphens}{url}
\PassOptionsToPackage{dvipsnames,svgnames,x11names}{xcolor}
%
\documentclass[
  a4paper,
  DIV=11,
  numbers=noendperiod]{scrartcl}

\usepackage{amsmath,amssymb}
\usepackage{lmodern}
\usepackage{iftex}
\ifPDFTeX
  \usepackage[T1]{fontenc}
  \usepackage[utf8]{inputenc}
  \usepackage{textcomp} % provide euro and other symbols
\else % if luatex or xetex
  \usepackage{unicode-math}
  \defaultfontfeatures{Scale=MatchLowercase}
  \defaultfontfeatures[\rmfamily]{Ligatures=TeX,Scale=1}
\fi
% Use upquote if available, for straight quotes in verbatim environments
\IfFileExists{upquote.sty}{\usepackage{upquote}}{}
\IfFileExists{microtype.sty}{% use microtype if available
  \usepackage[]{microtype}
  \UseMicrotypeSet[protrusion]{basicmath} % disable protrusion for tt fonts
}{}
\makeatletter
\@ifundefined{KOMAClassName}{% if non-KOMA class
  \IfFileExists{parskip.sty}{%
    \usepackage{parskip}
  }{% else
    \setlength{\parindent}{0pt}
    \setlength{\parskip}{6pt plus 2pt minus 1pt}}
}{% if KOMA class
  \KOMAoptions{parskip=half}}
\makeatother
\usepackage{xcolor}
\setlength{\emergencystretch}{3em} % prevent overfull lines
\setcounter{secnumdepth}{5}
% Make \paragraph and \subparagraph free-standing
\ifx\paragraph\undefined\else
  \let\oldparagraph\paragraph
  \renewcommand{\paragraph}[1]{\oldparagraph{#1}\mbox{}}
\fi
\ifx\subparagraph\undefined\else
  \let\oldsubparagraph\subparagraph
  \renewcommand{\subparagraph}[1]{\oldsubparagraph{#1}\mbox{}}
\fi


\providecommand{\tightlist}{%
  \setlength{\itemsep}{0pt}\setlength{\parskip}{0pt}}\usepackage{longtable,booktabs,array}
\usepackage{calc} % for calculating minipage widths
% Correct order of tables after \paragraph or \subparagraph
\usepackage{etoolbox}
\makeatletter
\patchcmd\longtable{\par}{\if@noskipsec\mbox{}\fi\par}{}{}
\makeatother
% Allow footnotes in longtable head/foot
\IfFileExists{footnotehyper.sty}{\usepackage{footnotehyper}}{\usepackage{footnote}}
\makesavenoteenv{longtable}
\usepackage{graphicx}
\makeatletter
\def\maxwidth{\ifdim\Gin@nat@width>\linewidth\linewidth\else\Gin@nat@width\fi}
\def\maxheight{\ifdim\Gin@nat@height>\textheight\textheight\else\Gin@nat@height\fi}
\makeatother
% Scale images if necessary, so that they will not overflow the page
% margins by default, and it is still possible to overwrite the defaults
% using explicit options in \includegraphics[width, height, ...]{}
\setkeys{Gin}{width=\maxwidth,height=\maxheight,keepaspectratio}
% Set default figure placement to htbp
\makeatletter
\def\fps@figure{htbp}
\makeatother

\KOMAoption{captions}{tableheading}
\makeatletter
\makeatother
\makeatletter
\makeatother
\makeatletter
\@ifpackageloaded{caption}{}{\usepackage{caption}}
\AtBeginDocument{%
\ifdefined\contentsname
  \renewcommand*\contentsname{Table of contents}
\else
  \newcommand\contentsname{Table of contents}
\fi
\ifdefined\listfigurename
  \renewcommand*\listfigurename{List of Figures}
\else
  \newcommand\listfigurename{List of Figures}
\fi
\ifdefined\listtablename
  \renewcommand*\listtablename{List of Tables}
\else
  \newcommand\listtablename{List of Tables}
\fi
\ifdefined\figurename
  \renewcommand*\figurename{Figure}
\else
  \newcommand\figurename{Figure}
\fi
\ifdefined\tablename
  \renewcommand*\tablename{Table}
\else
  \newcommand\tablename{Table}
\fi
}
\@ifpackageloaded{float}{}{\usepackage{float}}
\floatstyle{ruled}
\@ifundefined{c@chapter}{\newfloat{codelisting}{h}{lop}}{\newfloat{codelisting}{h}{lop}[chapter]}
\floatname{codelisting}{Listing}
\newcommand*\listoflistings{\listof{codelisting}{List of Listings}}
\makeatother
\makeatletter
\@ifpackageloaded{caption}{}{\usepackage{caption}}
\@ifpackageloaded{subcaption}{}{\usepackage{subcaption}}
\makeatother
\makeatletter
\@ifpackageloaded{tcolorbox}{}{\usepackage[many]{tcolorbox}}
\makeatother
\makeatletter
\@ifundefined{shadecolor}{\definecolor{shadecolor}{rgb}{.97, .97, .97}}
\makeatother
\makeatletter
\makeatother
\ifLuaTeX
  \usepackage{selnolig}  % disable illegal ligatures
\fi
\IfFileExists{bookmark.sty}{\usepackage{bookmark}}{\usepackage{hyperref}}
\IfFileExists{xurl.sty}{\usepackage{xurl}}{} % add URL line breaks if available
\urlstyle{same} % disable monospaced font for URLs
\hypersetup{
  pdftitle={Output Control at BPLIM},
  pdfauthor={Banco de Portugal's Microdata Research Laboratory (BPLIM)},
  colorlinks=true,
  linkcolor={blue},
  filecolor={Maroon},
  citecolor={Blue},
  urlcolor={Blue},
  pdfcreator={LaTeX via pandoc}}

\title{Output Control at BPLIM}
\author{Banco de Portugal's Microdata Research Laboratory (BPLIM)}
\date{2/22/24}

\begin{document}
\maketitle
\ifdefined\Shaded\renewenvironment{Shaded}{\begin{tcolorbox}[breakable, frame hidden, interior hidden, borderline west={3pt}{0pt}{shadecolor}, sharp corners, boxrule=0pt, enhanced]}{\end{tcolorbox}}\fi

\hypertarget{rules-for-outputs-extraction-at-bplim}{%
\section{Rules for Outputs Extraction at
BPLIM}\label{rules-for-outputs-extraction-at-bplim}}

Only researchers identified in the project may request extraction of
outputs. BPLIM will not check the ``correctness'' of the scripts used to
produce the output files. The code is the sole responsibility of the
researcher. However, researchers should abide by the following rules:

\begin{quote}
\begin{itemize}
\tightlist
\item
  Output files should never contain information that reveals individual
  records. This means that listings of individual records, tables with
  cells whose values were obtained by manipulation of three or less
  observations, statistical measures with standard errors of zero,
  minimum and maximum values, etc., are not allowed. BPLIM may refuse
  disclosure of output files if it perceives that the information in the
  logs may, directly or indirectly, reveal confidential information.
\end{itemize}
\end{quote}

\begin{quote}
\begin{itemize}
\tightlist
\item
  All outputs files must be generated by a script file which should be
  easy to identify. Requests for output extraction may be refused if
  BPLIM cannot associate the script file with the output.
\end{itemize}
\end{quote}

\begin{quote}
\begin{itemize}
\tightlist
\item
  All aggregate statistics must report the underlying value of N. This
  means, for example, that all regression outputs must report the number
  of observations and tables must report the number of observations per
  cell. Depending on the type of data BPLIM may impose stricter criteria
  at its discretion.
\end{itemize}
\end{quote}

\begin{quote}
\begin{itemize}
\tightlist
\item
  Output files of results must be plain text files (allowed formats are
  ``txt'', ``csv'', ``log'' and ``tex'').
\end{itemize}
\end{quote}

\begin{quote}
\begin{itemize}
\tightlist
\item
  Comments in output files should not include references to data values.
\end{itemize}
\end{quote}

\begin{quote}
\begin{itemize}
\tightlist
\item
  The only graphical outputs allowed are ``png'' files. The information
  depicted in the graph must be of aggregated values. In some
  circumstances BPLIM may request that authors report a table with the N
  associated with each data point depicted in the graphic.
\end{itemize}
\end{quote}

\begin{quote}
\begin{itemize}
\tightlist
\item
  Researchers should keep their output requests to a minimum and, as
  much as possible, these should be of final outputs.
\end{itemize}
\end{quote}

Note that output verification depends on staff availability and may take
longer in periods when the workload is higher. BPLIM staff will only
answer requests for output extraction sent by email and will take as
long as needed to ensure that all confidentiality requirements are
safeguarded.

To request an extraction researchers should place the files in the
``results'' folder and send an email requesting extraction of the
results to \textbf{bplim@bportugal.pt}. Please do not forget to include
the project number in the subject.



\end{document}
