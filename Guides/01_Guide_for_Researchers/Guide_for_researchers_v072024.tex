% Options for packages loaded elsewhere
\PassOptionsToPackage{unicode}{hyperref}
\PassOptionsToPackage{hyphens}{url}
\PassOptionsToPackage{dvipsnames,svgnames,x11names}{xcolor}
%
\documentclass[
  a4paper,
  DIV=11,
  numbers=noendperiod]{scrartcl}

\usepackage{amsmath,amssymb}
\usepackage{lmodern}
\usepackage{iftex}
\ifPDFTeX
  \usepackage[T1]{fontenc}
  \usepackage[utf8]{inputenc}
  \usepackage{textcomp} % provide euro and other symbols
\else % if luatex or xetex
  \usepackage{unicode-math}
  \defaultfontfeatures{Scale=MatchLowercase}
  \defaultfontfeatures[\rmfamily]{Ligatures=TeX,Scale=1}
\fi
% Use upquote if available, for straight quotes in verbatim environments
\IfFileExists{upquote.sty}{\usepackage{upquote}}{}
\IfFileExists{microtype.sty}{% use microtype if available
  \usepackage[]{microtype}
  \UseMicrotypeSet[protrusion]{basicmath} % disable protrusion for tt fonts
}{}
\makeatletter
\@ifundefined{KOMAClassName}{% if non-KOMA class
  \IfFileExists{parskip.sty}{%
    \usepackage{parskip}
  }{% else
    \setlength{\parindent}{0pt}
    \setlength{\parskip}{6pt plus 2pt minus 1pt}}
}{% if KOMA class
  \KOMAoptions{parskip=half}}
\makeatother
\usepackage{xcolor}
\setlength{\emergencystretch}{3em} % prevent overfull lines
\setcounter{secnumdepth}{5}
% Make \paragraph and \subparagraph free-standing
\ifx\paragraph\undefined\else
  \let\oldparagraph\paragraph
  \renewcommand{\paragraph}[1]{\oldparagraph{#1}\mbox{}}
\fi
\ifx\subparagraph\undefined\else
  \let\oldsubparagraph\subparagraph
  \renewcommand{\subparagraph}[1]{\oldsubparagraph{#1}\mbox{}}
\fi


\providecommand{\tightlist}{%
  \setlength{\itemsep}{0pt}\setlength{\parskip}{0pt}}\usepackage{longtable,booktabs,array}
\usepackage{calc} % for calculating minipage widths
% Correct order of tables after \paragraph or \subparagraph
\usepackage{etoolbox}
\makeatletter
\patchcmd\longtable{\par}{\if@noskipsec\mbox{}\fi\par}{}{}
\makeatother
% Allow footnotes in longtable head/foot
\IfFileExists{footnotehyper.sty}{\usepackage{footnotehyper}}{\usepackage{footnote}}
\makesavenoteenv{longtable}
\usepackage{graphicx}
\makeatletter
\def\maxwidth{\ifdim\Gin@nat@width>\linewidth\linewidth\else\Gin@nat@width\fi}
\def\maxheight{\ifdim\Gin@nat@height>\textheight\textheight\else\Gin@nat@height\fi}
\makeatother
% Scale images if necessary, so that they will not overflow the page
% margins by default, and it is still possible to overwrite the defaults
% using explicit options in \includegraphics[width, height, ...]{}
\setkeys{Gin}{width=\maxwidth,height=\maxheight,keepaspectratio}
% Set default figure placement to htbp
\makeatletter
\def\fps@figure{htbp}
\makeatother

\KOMAoption{captions}{tableheading}
\makeatletter
\makeatother
\makeatletter
\makeatother
\makeatletter
\@ifpackageloaded{caption}{}{\usepackage{caption}}
\AtBeginDocument{%
\ifdefined\contentsname
  \renewcommand*\contentsname{Table of contents}
\else
  \newcommand\contentsname{Table of contents}
\fi
\ifdefined\listfigurename
  \renewcommand*\listfigurename{List of Figures}
\else
  \newcommand\listfigurename{List of Figures}
\fi
\ifdefined\listtablename
  \renewcommand*\listtablename{List of Tables}
\else
  \newcommand\listtablename{List of Tables}
\fi
\ifdefined\figurename
  \renewcommand*\figurename{Figure}
\else
  \newcommand\figurename{Figure}
\fi
\ifdefined\tablename
  \renewcommand*\tablename{Table}
\else
  \newcommand\tablename{Table}
\fi
}
\@ifpackageloaded{float}{}{\usepackage{float}}
\floatstyle{ruled}
\@ifundefined{c@chapter}{\newfloat{codelisting}{h}{lop}}{\newfloat{codelisting}{h}{lop}[chapter]}
\floatname{codelisting}{Listing}
\newcommand*\listoflistings{\listof{codelisting}{List of Listings}}
\makeatother
\makeatletter
\@ifpackageloaded{caption}{}{\usepackage{caption}}
\@ifpackageloaded{subcaption}{}{\usepackage{subcaption}}
\makeatother
\makeatletter
\@ifpackageloaded{tcolorbox}{}{\usepackage[many]{tcolorbox}}
\makeatother
\makeatletter
\@ifundefined{shadecolor}{\definecolor{shadecolor}{rgb}{.97, .97, .97}}
\makeatother
\makeatletter
\makeatother
\ifLuaTeX
  \usepackage{selnolig}  % disable illegal ligatures
\fi
\IfFileExists{bookmark.sty}{\usepackage{bookmark}}{\usepackage{hyperref}}
\IfFileExists{xurl.sty}{\usepackage{xurl}}{} % add URL line breaks if available
\urlstyle{same} % disable monospaced font for URLs
\hypersetup{
  pdftitle={Guide for Researchers},
  pdfauthor={Banco de Portugal's Microdata Research Laboratory (BPLIM)},
  colorlinks=true,
  linkcolor={blue},
  filecolor={Maroon},
  citecolor={Blue},
  urlcolor={Blue},
  pdfcreator={LaTeX via pandoc}}

\title{Guide for Researchers}
\author{Banco de Portugal's Microdata Research Laboratory (BPLIM)}
\date{7/24/24}

\begin{document}
\maketitle
\ifdefined\Shaded\renewenvironment{Shaded}{\begin{tcolorbox}[borderline west={3pt}{0pt}{shadecolor}, breakable, enhanced, sharp corners, frame hidden, interior hidden, boxrule=0pt]}{\end{tcolorbox}}\fi

\renewcommand*\contentsname{Contents}
{
\hypersetup{linkcolor=}
\setcounter{tocdepth}{2}
\tableofcontents
}
\newpage{}

\hypertarget{general-remarks}{%
\section{General Remarks}\label{general-remarks}}

The Banco de Portugal Microdata Research Laboratory (BPLIM) is part of
the Economics and Research Department (DEE) of Banco de Portugal (BdP)
and was created with the objective of facilitating researchers' access
to and use of micro datasets about the Portuguese economy. Primarily,
these are datasets owned by BdP, but there are other datasets supplied
by third parties. Data can only be accessed on BPLIM's \textbf{External
Server} or in the internal infrastructure of the bank. The External
Server has \textbf{remote access} capabilities and is intended for use
by external researchers. Internal researchers (those with a contractual
link to BdP) have access to a scalable computational platform
(\textbf{Pitagoras}) and to a data repository created by BPLIM, which
can be used without restrictions.

\hypertarget{data-available-for-researchers}{%
\section{Data Available for
Researchers}\label{data-available-for-researchers}}

\hypertarget{what-are-the-characteristics-of-bplims-datasets}{%
\subsection{What are the characteristics of BPLIM's
datasets?}\label{what-are-the-characteristics-of-bplims-datasets}}

All BPLIM datasets created for research purposes are stripped of
elements that allow for direct identification of companies, banks, or
individuals. Whenever possible, the datasets contain \textbf{unique unit
identifiers} common across datasets: examples of these are \emph{tina}
-- the tax identification number anonymized for companies -- and
\emph{bina} -- the bank identification number anonymized.

By default BPLIM datasets are made available in \emph{Stata} format.
Larger datasets may be made available in \emph{parquet} format. Data is
stored in an efficient way that minimizes file size and follows BPLIM's
naming convention. \textbf{Labels} are applied to all variables and
value labels to all categorical variables. Whenever possible, labels can
be displayed in Portuguese and English.

All datasets are accompanied by a \textbf{Manual} that contains all
relevant information regarding the data. The data manuals, the
metafiles\footnote{All metafiles are created with the
  \href{https://github.com/BPLIM/Tools/tree/master/ados/General/metaxl}{\emph{metaxl}}
  Stata command.} and citation information for the different data
extractions are available on
\href{https://github.com/BPLIM/Manuals/tree/master/Data}{\textbf{GitHub}}.
A metafile that contains additional descriptive statistics for each
dataset can be obtained once researchers are given access to the server.
Please refer to the
\href{https://github.com/BPLIM/Manuals/tree/master/Guides/03_External_Server}{\textbf{External
Server Guide}} on how to access this information.

BPLIM datasets may also have companion script files that calculate
additional variables or harmonized variables to guarantee comparability
over time and across datasets. Datasets are updated regularly based on a
\textbf{data extraction} (``data freeze'') at a specific point in time,
and a versioning system is applied to reflect any changes to the data
set. Most datasets have an associated
\href{https://www.da-ra.de/dara/search/search_result?lang=en\&detail=false\&mdlang=en\&personal=false\&dsquery=BPLIM}{\textbf{Digital
Object Identifier}} (DOI).

\hypertarget{what-data-are-available}{%
\subsection{What data are available?}\label{what-data-are-available}}

The complete \textbf{list of datasets}, including a short description
and the access conditions, is available in the
\href{https://github.com/BPLIM/Manuals/tree/master/Guides/02_BPLIM_Datasets_Guide}{\textbf{BPLIM
Datasets Guide}}. On
\href{https://msites-dee-bplim-prd.azurewebsites.net}{\textbf{BPLIM´s
website}} you will find a list of the latest version of the datasets
available for external researchers, along with a link to the respective
documentation.

\hypertarget{data-access}{%
\section{Data Access}\label{data-access}}

\hypertarget{who-can-gain-access-to-the-data}{%
\subsection{Who can gain access to the
data?}\label{who-can-gain-access-to-the-data}}

Access is restricted to BPLIM accredited researchers who intend to
utilize the data for scientific purposes. Individuals affiliated with
BdP are classified as \textbf{Internal Researchers} and have
unrestricted access to all datasets maintained by BPLIM. Those not
affiliated with BdP are considered \textbf{External Researchers}, and
their access is subject to several restrictions.
\href{https://github.com/BPLIM/Manuals/tree/master/Guides/02_BPLIM_Datasets_Guide}{\textbf{BPLIM
Datasets Guide}} summarizes the access restrictions for each dataset.

\hypertarget{how-can-researchers-request-access-to-the-data}{%
\subsection{How can researchers request access to the
data?}\label{how-can-researchers-request-access-to-the-data}}

\textbf{Internal researchers} are provided access to a data repository
maintained by BPLIM. The repository makes available data, which can be
used unreservedly for research and policy activities without any
formality. However, data made available strictly for policy should not
be used for research because as it may have lower quality, are not
documented, and may not be reproducible. If Internal Researchers are
working with external co-authors or if they need to use or link other
micro datasets not available in the data repository they will need to
submit a project to BPLIM.

\textbf{External Researchers} must always submit a project. The
\textbf{project proposal} must: (1) contain a short description of the
research project; (2) identify all participants involved in the project
along with their affiliations, and include a curriculum vitae (CV) for
each; and (3) specify the datasets, timeframe, and variables required.
All external researchers with access to the data are required to sign a
\emph{confidentiality agreement}. If the project consists of a master or
doctoral dissertation then the supervisor(s) has to be identified and
must also sign the \emph{confidentiality agreement}. BPLIM staff can
collaborate with the researcher(s) to identify the required datasets
and, if necessary, construct a customized dataset. A copy of the
required documents can be found in
\href{https://msites-dee-bplim-prd.azurewebsites.net/content/access-0}{\textbf{BPLIM's
website}}.

\hypertarget{approval-process}{%
\subsection{Approval process}\label{approval-process}}

Upon submission of all required documentation and verification that it
conforms to BPLIM rules, the project will be evaluated to ensure that it
addresses a legitimate research question. \textbf{Compliance with the
guidelines} is crucial for recurring external researchers (those who
have already participated in BPLIM projects). Once the project is
approved, the researcher will receive an email notification with the
\textbf{user credentials and instructions} for accessing the data.
Summary information about the project and researchers will be posted on
BPLIM's website.

\hypertarget{how-can-the-data-be-accessed}{%
\subsection{How can the data be
accessed?}\label{how-can-the-data-be-accessed}}

When applying for a project, researchers must specify if they plan to
access an internal account, in \emph{Pitagoras}, or an external account
in the External Server.

Accounts open at \textbf{\emph{Pitagoras}} can only be accessed at the
installations of BdP (\textbf{``on-site access''}) either at Lisbon or
Porto. Internal researchers can log into \emph{Pitagoras} from their
terminal using their network login credentials. External researchers
will be provided with a login and password for \emph{Pitagoras} and
granted access to a terminal where it is technically restricted to
transfer, download, copy, paste, or print any data. BPLIM projects at
\emph{Pitagoras} are placed in a specific folder containing all
projects, with users having access only to their designated project
folder.

For more details on accessing BPLIM projects in \emph{Pitagoras}, please
refer to the \emph{BPLIM Pitagoras Manual}. Due to a limited number of
terminals available, external researchers must book their visits well in
advance.

If the account is on the \textbf{External Server}, then the data must be
accessed remotely (\textbf{``remote access''}) using a secure
connection. BPLIM uses the
\href{https://www.nomachine.com/}{\textbf{\emph{NoMachine}}} software
for this purpose. With this connection, it is not possible to exchange
files between the external server and the local computer. For more
details on using the External Server, please refer to the
\href{https://github.com/BPLIM/Manuals/tree/master/Guides/03_External_Server}{\textbf{External
Server Guide}}.

In special circumstances explained below, the external researcher may be
granted indirect access to the data (\textbf{``surrogate access''}, also
known as, \textbf{``remote execution''}). With surrogate access, there
is no need for the external researcher to have an account, as BPLIM
staff (or an internal researcher) will act as a proxy for data access.
This means that BPLIM staff (or an internal researcher) will execute the
scripts written by the external researcher and share the outputs after
disclosure control.

\hypertarget{what-type-of-anonymization-is-applied-to-bplims-datasets}{%
\subsection{What type of anonymization is applied to BPLIM's
datasets?}\label{what-type-of-anonymization-is-applied-to-bplims-datasets}}

When BPLIM makes its datasets accessible to researchers, it uses several
different strategies to anonymize the data. The type of anonymization
depends on the specific data and the user. BPLIM uses \textbf{four
levels of anonymization}:

\begin{itemize}
\item
  \textbf{Level 1} - All information that could lead to the direct
  identification of statistical units (firms/banks/individuals) is
  omitted, and unique identifiers (e.g., NIF, bank ID) undergo a 1-to-1
  transformation to new identifiers that are specific to the project.
  Level 1 datasets will contain \textbf{``\_A\_''} in the name.
\item
  \textbf{Level 2} - in addition to Level 1, the values of variables
  containing sensitive information will be replace by modified values,
  which are random values that exhibit some correlation with the
  original values. The file name of a Level 2 dataset will contain
  \textbf{``\_P\_''}, and the labels of the modified variables will
  reflect this information.
\item
  \textbf{Level 3} - in addition to Level 2, variables may be sorted
  randomly and independently to break the link between the observations.
  Level 3 datasets will be identified with \textbf{``\_R\_''}.
\item
  \textbf{Level 4} - a subset of the data is generated randomly
  (pseudo-data), respecting only the metadata and the time structure of
  the original data. Level 4 datasets will be identified with
  \textbf{``\_D\_''} in their name.
\end{itemize}

Datasets of \textbf{Level 2, 3, or 4} are generically designated as
\textbf{modified} datasets.

\textbf{Level 4} datasets are the only ones that researchers are allowed
to use \textbf{outside of the bank computing environment}, because the
values generated for this level are fictitious.

\hypertarget{what-determines-the-type-of-anonymization-applied-to-the-data-set}{%
\subsection{What determines the type of anonymization applied to the
data
set?}\label{what-determines-the-type-of-anonymization-applied-to-the-data-set}}

BPLIM data meant to be used by \textbf{Internal researchers} are always
anonymized at Level 1. The exception is if the Internal Researcher is
accessing the data through the External Server. In that case, Internals
Researchers have the same access conditions as External Researchers.

Datasets made available to \textbf{External Researchers} are subject to
a \textbf{confidentiality classification} as follows: \textbf{low},
\textbf{medium}, or \textbf{high}. If the data are classified as
\textbf{low}, then the data is anonymized at Level 1. Datasets
classified as \textbf{medium} may be anonymized at Level 2 or 3,
depending on the risk of identification. For datasets classified with a
\textbf{high} level of confidentiality, External Researchers may only
have access to Level 4 data.

\hypertarget{how-can-researchers-work-with-modified-data}{%
\subsection{\texorpdfstring{How can researchers work with
\emph{modified}
data?}{How can researchers work with modified data?}}\label{how-can-researchers-work-with-modified-data}}

\textbf{Modified} datasets serve only the purpose of facilitating the
creation of scripts that manipulate/analyze the data. Results of
analysis performed on ``modified'' datasets \textbf{are not valid for
research purposes}. However, external researchers can always request to
have their scripts run on the original datasets. This rule applies
whether the access is ``on-site'' or ``remote''. Researchers working
with \textbf{modified} data should use BPLIM's \textbf{Replication App}.
For instructions on how to use the Replication App, please refer to the
\href{https://github.com/BPLIM/ReplicationApp/blob/main/Server/UserGuide.md}{\textbf{Replication
App User Guide}}. While not strictly enforced, use of the BPLIM's
Replication App, will ensure that the replication on the original data
is implemented correctly and in a much more timely manner. We strongly
encourage use of the BPLIM's Replication App.

Researchers working with Level 4 data (\textbf{pseudo-data}) in their
personal computers, will receive a package along with instructions to
create the pseudo-data. For instructions on how to work with
pseudo-data, please refer to the
\href{https://github.com/BPLIM/Manuals/tree/master/Guides/07_Working_with_Pseudo-Data}{\textbf{BPLIM
Guide to Working with Pseudo-Data}}.

\hypertarget{how-are-projects-involving-co-authorship-between-internal-researchers-and-external-researchers-handled}{%
\subsection{How are projects involving co-authorship between Internal
Researchers and External Researchers
handled?}\label{how-are-projects-involving-co-authorship-between-internal-researchers-and-external-researchers-handled}}

Projects where internal and external researchers have access to the data
are designated \textbf{``mixed projects''}.

If the mixed project is implemented in the \textbf{External Server} and
only data with low level of confidentiality is used, then the
distinction is irrelevant as all researchers are treated as external and
the data is anonymized (Level 1).

However, if the data needed for the project is classified at a higher
confidentiality level, external researchers can only access
\textbf{modified data} or, in the most restrictive cases,
\textbf{pseudo-data}, but never the original data. In such mixed
projects, where the external researcher does not have access to the
original, the internal researcher is responsible for ensuring that the
information shared with their external co-authors complies with the
confidentiality requirements associated with the data.

In these cases, BPLIM will open a second \textbf{``parallel'' account}
with access only for the internal researcher(s) and place all the
original (anonymized) data there. It will be the responsibility of the
internal researcher to execute all scripts on the original data stored
in this ``parallel'' account.

Furthermore, it will be their responsibility to ensure that external
researchers do not have any access to confidential data. Specifically,
external co-authors must not access the project account containing the
original data, the internal co-author's desktop computer, or any logs
that may contain confidential information.

In a mixed project all interaction with BPLIM should be done via the
internal researcher.

\hypertarget{how-can-an-external-researcher-gain-surrogate-access-to-a-dataset}{%
\subsection{How can an External Researcher gain surrogate access to a
dataset?}\label{how-can-an-external-researcher-gain-surrogate-access-to-a-dataset}}

The \emph{BPLIM Dataset Guide} lists the datasets that can be accessed
in surrogate mode by an external researcher that \textbf{does not have
an internal co-author}. In that case the external researcher needs to
submit a detailed project to BPLIM. The project will be evaluated
\textbf{according to the relevance of the topic} to the research agenda
of BdP. Only projects deemed relevant will be granted surrogate access.
If BPLIM decides to support the project, external researchers will be
assigned a data expert at BPLIM, who will collaborate with them to
prepare and run scripts on the original data. However, the coding itself
remains the ultimate responsibility of the external researcher, and
BPLIM will not validate or certify the scripts written by the
researcher. External researchers are encouraged to work closely with
BPLIM staff to ensure they achieve the intended results and are also
encouraged to discuss their findings with BPLIM staff. It is highly
recommended that external researchers initiate their project with a
short-stay visit in BPLIM, during which time they can discuss their
research with BPLIM staff and gain a thorough understanding of the data
complexities. Additional visits throughout the project are also
encouraged. All outputs shared with the external researcher are subject
to the usual disclosure restrictions.

\hypertarget{transfer-of-external-files}{%
\subsection{Transfer of external
files}\label{transfer-of-external-files}}

\textbf{Internal researchers} working in \emph{Pitagoras} can freely
copy files to and from their accounts. Thus, they are free to place
external files in their \emph{Pitagoras} accounts. However, if the
external data needs to be merged with BPLIM datasets using an anonymized
key, then the internal researcher must be working in a BPLIM project
account at \emph{Pitagoras}. They will also need to fill in an
application for using an external dataset (see below). BPLIM will
anonymize the external datasets using the same linking key as the one
used for the BPLIM datasets. Note that BPLIM identifiers (eg:
\emph{tina} and \emph{bina}) are specific to a project and are not valid
to link files exchanged between accounts.

If the account is shared with external researchers - a \textbf{``mixed
project''} - the internal researcher must ensure that external
researchers are allowed access to the external dataset and that they do
not gain undue access to confidential data. At the request of the
internal researcher, BPLIM will anonymize/modify the external datasets
intended for use in ``mixed projects''.

\textbf{External researchers} may also request that \textbf{external
data files} be placed in their accounts. BPLIM staff will assist if
there is a need to merge external datasets with BPLIM datasets. External
datasets typically contain aggregated data, but it may be possible to
add external datasets with finer granularity. BPLIM staff will assess if
the addition of the external datasets increases the risk of
identification of individual observations. In such cases, additional
measures will be undertaken to ensure that the confidentiality of the
data is preserved once external files are merged with existing BPLIM
datasets. These situations will be analyzed case by case and discussed
with BPLIM staff.

All external datasets provided to BPLIM should be in a \textbf{Stata or
CSV format} and an
\href{https://msites-dee-bplim-prd.azurewebsites.net/sites/default/files/external_dataset_application_vf.pdf}{\textbf{External
Datasets Form}} must be filled in. In the form, researchers are required
to explain the data provenance, provide a justification for its use, and
identify the key variables that enable linking the external dataset with
BPLIM's datasets. The researcher must also certify that all researchers
with access to the account are authorized to use the data. It is the
responsibility of the researcher to ensure the external files can be
legitimately used for that purpose.

\hypertarget{statistical-software}{%
\section{Statistical Software}\label{statistical-software}}

\hypertarget{statistical-software-1}{%
\subsection{Statistical Software}\label{statistical-software-1}}

When researchers apply for a new project, they will need to specify the
software to be used. Available options are \textbf{Stata}, \textbf{R},
\textbf{Julia}, and \textbf{Python}. BPLIM provides the researcher with
a default list of \textbf{external packages/ados} for each software. If
researchers require additional packages, they must specify the package,
its source, and version.

For each project, BPLIM creates a \textbf{container} with the software
and packages required for the project. If researchers require
\textbf{additional external packages} during the project, they should
send a \textbf{request via email} to BPLIM.

Once the project account is set up, researchers will have access to a
\textbf{Linux environment} where they can use the container.
\textbf{Templates} for writing code are also provided in the account,
and researchers should strive to adhere to the conventions outlined in
these templates as much as possible.

\hypertarget{bplim-tools}{%
\subsection{BPLIM Tools}\label{bplim-tools}}

BPLIM staff has developed several \textbf{Stata packages} to assist
researchers in their tasks. Some of these tools are tailored for use
with BPLIM datasets, while others have broader utility. To promote
transparency in coding and versioning, BPLIM makes all tools available
on \href{https://github.com/BPLIM/Tools}{\textbf{Github}}. Users are
welcome to suggest improvements or add their own contributions. Tools
with general applicability can be installed directly from Github on any
internet-connected machine.

\hypertarget{output-extraction}{%
\section{Output Extraction}\label{output-extraction}}

\hypertarget{can-researchers-transfer-files-from-the-server}{%
\subsection{Can researchers transfer files from the
server?}\label{can-researchers-transfer-files-from-the-server}}

External researchers are \textbf{never allowed to transfer files to/from
BPLIM's accounts}. This policy also applies to internal researchers
accessing the external server. However, internal researchers have the
flexibility to transfer files to and from their accounts in
\emph{Pitagoras}, including data and output logs.

When internal researchers collaborate with external co-authors, it
becomes their responsibility to ensure that all files shared with
external co-authors comply with BPLIM's data security and
confidentiality policies. This ensures that no sensitive or restricted
information is improperly disseminated.

For further details, please refer to the
\href{https://github.com/BPLIM/Manuals/blob/master/Guides/06_Output_Control}{\textbf{Output
Control Guide}}.

\hypertarget{what-restrictions-apply-to-the-release-of-output-logs}{%
\subsection{What restrictions apply to the release of output
logs?}\label{what-restrictions-apply-to-the-release-of-output-logs}}

As a general principle, BPLIM will not verify the ``correctness'' of
scripts used by researchers to generate logs and expressly disclaims
responsibility for any errors or inaccuracies in researchers' code. This
responsibility solely rests with the researcher.

Output logs should never contain information that discloses individual
dataset records; they should only include \textbf{aggregate-level
information}. Therefore, listings of individual records, tables with
cells derived from manipulation of three or fewer observations,
statistical measures with standard errors of zero, minimum and maximum
values, etc., are prohibited.

\textbf{Plain text files} (including Latex and comma or tab separated
values) are the preferable formats, although other formats may be
accepted provided that the data content can be easily verified.
\textbf{Graphical outputs} should be generated in ``.png'' format.

In mixed projects, the internal researcher is responsible for ensuring
adherence to these principles. BPLIM will assist in this process upon
request from the internal researcher. For projects involving only
external researchers, BPLIM staff will verify the conformity of all
outputs.

\textbf{Output disclosure control} depends on staff availability and may
take longer in periods of high workload. BPLIM staff will only answer
requests for \textbf{output extraction sent by email} and will take the
necessary time to ensure that all \textbf{confidentiality requirements
are safeguarded}. Researchers should keep their output requests to a
minimum and, whenever feasible, these requests should be of
\textbf{final outputs}.

For further details, please refer to the
\href{https://github.com/BPLIM/Manuals/blob/master/Guides/06_Output_Control}{\textbf{Output
Control Guide}}.

\hypertarget{what-happens-if-the-researcher-violates-the-rules}{%
\subsection{What happens if the researcher violates the
rules?}\label{what-happens-if-the-researcher-violates-the-rules}}

BPLIM assumes that all researchers operate in good faith and should
endeavor to adhere to the provisions outlined in the signed
\textbf{``Declaration on Confidentiality and Use of Data''}. In cases
where a researcher engages in behavior deemed inappropriate, BPLIM
reserves the right to terminate or suspend all projects involving the
researcher and the institution to which he/she is affiliated.

\hypertarget{replicability-and-data-archiving}{%
\section{Replicability and Data
Archiving}\label{replicability-and-data-archiving}}

\hypertarget{replicability-of-work}{%
\subsection{Replicability of work}\label{replicability-of-work}}

Researchers working at BPLIM have all the necessary conditions to ensure
that their work is reproducible. All \textbf{BPLIM datasets are
versioned} and can be exactly recreated based on archived extractions.
Researchers have access to a \textbf{Singularity container} specific to
their project, including the respective definition file. This means that
the computing enviroment is also replicable. All external files,
including external datasets and scripts, are stored in the project
folder. Researchers can utilize BPLIM's \textbf{Replication App} to
verify the replicability of results and to generate a
\textbf{replication package}. Ultimately, it is up to the researcher to
garantee that his/her work is reproducible.

\hypertarget{replicability-by-third-parties}{%
\subsection{Replicability by third
parties}\label{replicability-by-third-parties}}

If requested, BPLIM will work with third-parties such as \textbf{data
editors}, \textbf{certification services}, or \textbf{individual
researchers} to provide conditions for replication of results of
individual projects.

The replication process consists of opening an account for the
``replicator'' and providing him/her with access to the same conditions
as the researcher(s). Projects by internal researchers may have to be
replicated in ``surrogate'' mode.

To \textbf{facilitate replication}, researchers should unequivocally
identify all datasets used in the analysis, ideally using the
\textbf{Replication App}. All other needed files should be provided by
the ``replicator''.

BPLIM will work directly with data editors or certification services
(e.g.~Cascad) to evaluate the best approach to implement their
replication protocol. In the case of individual researchers willing to
act as ``replicators'' they will have to go through the standard process
of submitting a research project.

\hypertarget{archiving}{%
\subsection{Archiving}\label{archiving}}

By default, once a project is closed BPLIM will keep a \textbf{copy of
all syntax files} (e.g.~text files with ``do'', ``R'', ``py'', and
``jl'' extensions) plus all files found in the
``\emph{initial\_dataset}'' and the ``\emph{tools}'' folders.

A copy of the syntax files will be sent to the researcher, who should
verify that the list is complete. All other files will be deleted unless
the researcher explicitly requests that certain external file(s) are
archived. Ideally, the researcher should use the Replication app and
\textbf{save the replication package} created by the application.

Only in exceptional and well justified circumstances will BPLIM agree to
archive intermediate data files created by the researchers. It is the
responsibility of the researcher to ensure that all files needed for
proper replication of the results are archived with the project.

Archives are kept for \textbf{ten years} since the closing of the
project. BPLIM may archive a project that has been \textbf{inactive for
more than one year}.

\hypertarget{publication-of-research-papers}{%
\section{Publication of Research
Papers}\label{publication-of-research-papers}}

\hypertarget{citations-and-research-outputs}{%
\subsection{Citations and research
outputs}\label{citations-and-research-outputs}}

All BPLIM datasets should be cited according to the information provided
in the data manual. If available, the \textbf{Digital Object Identifier}
(DOI) should be referenced.

When the topic analyzed by the external researcher(s) bears special
relationship to BdP's statutory tasks, researchers are encouraged to
\textbf{discuss their results} with BdP staff. In this case, BdP may ask
to see a copy of the work, prior to any public release, and may provide
suggestions regarding the research project.

Moreover, in situations where data access is granted based on the
relevance of the topic (\textbf{``surrogate access''}) researchers are
not only encouraged to discuss their results with BdP staff but must
also seek BdP \textbf{approval prior to any public release} of their
results.

As soon as available, researchers are \textbf{required to send to BPLIM
a copy of all research outputs} (working paper, conference proceedings,
paper, thesis, etc.) related to the project.

\hypertarget{contacting-bplim}{%
\section{Contacting BPLIM}\label{contacting-bplim}}

\hypertarget{how-can-i-contact-bplim}{%
\subsection{How can I contact BPLIM?}\label{how-can-i-contact-bplim}}

The preferred way to contact BPLIM is through email:
\href{mailto:bplim@bportugal.pt}{\textbf{BPLIM@bportugal.pt}}. For
projects already ongoing the subject line should always include the
project reference (eg: p\#\#\#\_Surname). If necessary you can contact
us at:

\textbf{Address:}

Banco de Portugal

Microdata Research Laboratory

Rua do Almada, 71

4050-036 Porto

Portugal



\end{document}
